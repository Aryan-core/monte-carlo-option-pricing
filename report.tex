\documentclass[11pt]{article}

% ---------------- Packages ----------------
\usepackage{amsmath,amssymb,amsthm}
\usepackage{mathtools}
\usepackage{bm}
\usepackage{geometry}
\usepackage{hyperref}
\usepackage{graphicx}
\usepackage{booktabs}
\usepackage{caption}
\usepackage{float}
\usepackage{enumitem}

\geometry{margin=1in}
\hypersetup{
  colorlinks=true,
  linkcolor=blue,
  urlcolor=blue,
  citecolor=blue
}

% ---------------- Theorem Environments ----------------
\theoremstyle{plain}
\newtheorem{theorem}{Theorem}[section]
\newtheorem{proposition}[theorem]{Proposition}

\theoremstyle{definition}
\newtheorem{definition}[theorem]{Definition}

\theoremstyle{remark}
\newtheorem{remark}[theorem]{Remark}

% ---------------- Commands ----------------
\newcommand{\E}{\mathbb{E}}
\newcommand{\Var}{\mathrm{Var}}
\newcommand{\Prob}{\mathbb{P}}
\newcommand{\R}{\mathbb{R}}
\newcommand{\1}{\mathbf{1}}
\newcommand{\normcdf}{\Phi}

% ---------------- Title ----------------
\title{A Monte Carlo Option Pricing Engine: \\
Risk-Neutral GBM, Error Analysis, and Variance Reduction}
\author{Aryan Khan}
\date{\today}

\begin{document}
\maketitle
\tableofcontents
\newpage

% ============================================================
\begin{abstract}
We develop a Monte Carlo pricing engine for European options under the Black--Scholes model.
We simulate stock price paths using the exact solution of geometric Brownian motion (GBM) under the
risk-neutral measure and estimate discounted payoffs. We present a statistical error analysis via the
central limit theorem, construct confidence intervals, and empirically verify the $O(N^{-1/2})$
convergence rate. We also implement antithetic variates as a variance reduction technique and compare
Monte Carlo prices to the Black--Scholes closed-form solution. The final result is a clean, reproducible
pricing pipeline suitable for extension to more advanced derivatives.
\end{abstract}

% ============================================================
\section{Introduction}
Monte Carlo simulation is a general-purpose numerical method for pricing derivatives when closed-form
formulas are unavailable or when payoffs depend on complex path features. Even in settings where an
analytic benchmark exists (such as Black--Scholes European options), Monte Carlo provides a valuable
testbed for understanding estimator variance, confidence intervals, and variance reduction methods.

\paragraph{Goals.}
\begin{enumerate}[label=\arabic*.]
\item Specify the Black--Scholes model and its risk-neutral formulation.
\item Implement Monte Carlo estimators for European call and put options.
\item Quantify sampling error using standard errors and confidence intervals.
\item Demonstrate convergence behavior as the number of samples increases.
\item Improve efficiency using antithetic variates and measure variance reduction.
\end{enumerate}

% ============================================================
\section{Model: GBM and Risk-Neutral Pricing}
\subsection{Geometric Brownian Motion}
\begin{definition}[GBM under the physical measure]
A stock price process $(S_t)_{t\ge 0}$ follows geometric Brownian motion if
\[
dS_t = \mu S_t\,dt + \sigma S_t\,dW_t,
\]
where $\mu\in\R$ is the drift, $\sigma>0$ is the volatility, and $(W_t)$ is standard Brownian motion.
\end{definition}

\begin{definition}[Risk-neutral measure]
Under the risk-neutral measure, the discounted stock price becomes a martingale and the SDE takes the form
\[
dS_t = r S_t\,dt + \sigma S_t\,dW_t,
\]
where $r$ is the continuously-compounded risk-free interest rate.
\end{definition}

\subsection{Exact Sampling of $S_T$}
The GBM SDE admits a closed-form solution. Under the risk-neutral model,
\begin{equation}
S_T = S_0 \exp\left(\left(r-\tfrac{1}{2}\sigma^2\right)T + \sigma \sqrt{T}\,Z\right),
\qquad Z\sim \mathcal{N}(0,1).
\label{eq:gbm_exact}
\end{equation}
This exact sampling avoids discretization bias for European payoffs depending only on $S_T$.

% ============================================================
\section{European Option Pricing}
\subsection{Payoffs}
Let $K>0$ be the strike and $T>0$ the maturity. The European call and put payoffs are
\[
\text{Call payoff: } (S_T - K)^+,\qquad
\text{Put payoff: } (K - S_T)^+,
\]
where $x^+=\max\{x,0\}$.

\subsection{Risk-Neutral Valuation}
Under standard no-arbitrage assumptions, the time-$0$ price is the discounted risk-neutral expectation:
\begin{equation}
V_0 = e^{-rT}\E\big[\text{payoff}(S_T)\big].
\label{eq:rn_price}
\end{equation}

\subsection{Monte Carlo Estimator}
Let $S_T^{(1)},\dots,S_T^{(N)}$ be i.i.d.\ samples from \eqref{eq:gbm_exact}. Define the discounted payoff samples:
\[
Y_i := e^{-rT}\,\text{payoff}\big(S_T^{(i)}\big).
\]
Then the Monte Carlo estimator is
\begin{equation}
\widehat{V}_N := \frac{1}{N}\sum_{i=1}^N Y_i.
\label{eq:mc_estimator}
\end{equation}

\begin{remark}
The estimator $\widehat{V}_N$ is unbiased in this setting because we sample $S_T$ exactly and take the correct
risk-neutral discounted payoff.
\end{remark}

% ============================================================
\section{Error Analysis: CLT, Standard Error, and Confidence Intervals}
\subsection{Central Limit Theorem}
Assuming $\Var(Y_1)<\infty$, the CLT gives
\[
\sqrt{N}\left(\widehat{V}_N - V_0\right)\ \Rightarrow\ \mathcal{N}\big(0,\Var(Y_1)\big).
\]
Hence the estimation error is typically on the order of $N^{-1/2}$.

\subsection{Standard Error and 95\% CI}
Let the sample variance be
\[
\widehat{\Var}(Y) = \frac{1}{N-1}\sum_{i=1}^N (Y_i - \widehat{V}_N)^2,
\]
and define the standard error
\begin{equation}
\mathrm{SE}(\widehat{V}_N) = \sqrt{\frac{\widehat{\Var}(Y)}{N}}.
\label{eq:se}
\end{equation}
An approximate $95\%$ confidence interval is
\begin{equation}
\widehat{V}_N \ \pm\ 1.96\cdot \mathrm{SE}(\widehat{V}_N).
\label{eq:ci95}
\end{equation}

\begin{remark}
Confidence intervals are crucial in practice: two Monte Carlo runs can produce slightly different point estimates,
but if their confidence intervals overlap, the results are consistent within sampling error.
\end{remark}

% ============================================================
\section{Analytic Benchmark: Black--Scholes Formula}
For validation, we compare Monte Carlo to the Black--Scholes closed-form price.

Define
\[
d_1 = \frac{\ln(S_0/K) + (r+\tfrac{1}{2}\sigma^2)T}{\sigma\sqrt{T}},
\qquad
d_2 = d_1 - \sigma\sqrt{T},
\]
and let $\normcdf(\cdot)$ denote the standard normal CDF.

\subsection{European Call}
\begin{equation}
C_{\mathrm{BS}} = S_0\normcdf(d_1) - K e^{-rT}\normcdf(d_2).
\label{eq:bs_call}
\end{equation}

\subsection{European Put}
\begin{equation}
P_{\mathrm{BS}} = K e^{-rT}\normcdf(-d_2) - S_0\normcdf(-d_1).
\label{eq:bs_put}
\end{equation}

% ============================================================
\section{Variance Reduction: Antithetic Variates}
A standard variance reduction method uses negatively correlated samples.

\subsection{Construction}
Sample $Z\sim\mathcal{N}(0,1)$ and also use $-Z$. Define
\[
S_T(Z) = S_0 \exp\left(\left(r-\tfrac{1}{2}\sigma^2\right)T + \sigma\sqrt{T}\,Z\right),
\quad
S_T(-Z) = S_0 \exp\left(\left(r-\tfrac{1}{2}\sigma^2\right)T - \sigma\sqrt{T}\,Z\right).
\]
For a payoff function $g(\cdot)$, define the paired estimator
\[
\widetilde{Y} = \frac{e^{-rT}}{2}\Big(g(S_T(Z)) + g(S_T(-Z))\Big).
\]
Then average i.i.d.\ copies of $\widetilde{Y}$ over $N$ pairs.

\begin{proposition}[Variance reduction intuition]
If $g(S_T(Z))$ and $g(S_T(-Z))$ are negatively correlated, then
\[
\Var(\widetilde{Y}) < \Var(Y),
\]
reducing the Monte Carlo standard error at fixed computational cost.
\end{proposition}

\begin{remark}
Antithetic variates are especially effective when the payoff is monotone in $S_T$ (e.g.\ European calls/puts),
since $Z$ and $-Z$ push $S_T$ in opposite directions.
\end{remark}

% ============================================================
\section{Numerical Experiments}
\subsection{Experimental Setup}
In experiments we fix:
\[
S_0,\ K,\ r,\ \sigma,\ T,
\]
and compute:
\begin{itemize}
\item Monte Carlo estimate $\widehat{V}_N$ and 95\% CI,
\item Black--Scholes benchmark,
\item estimator variance with and without antithetic variates,
\item convergence behavior as $N$ varies (e.g.\ $N\in\{10^3,10^4,10^5,10^6\}$).
\end{itemize}

\subsection{Convergence Rate Study}
We expect Monte Carlo error to behave like $O(N^{-1/2})$. A practical check is to record
\[
\big|\widehat{V}_N - V_{\mathrm{BS}}\big|
\]
and plot it against $N$ on a log-log scale; the slope should be close to $-1/2$.

\subsection{(Optional) Figures}
If you later generate plots using Python, place them under \texttt{figures/} and uncomment these.

% \begin{figure}[H]
% \centering
% \includegraphics[width=0.85\textwidth]{figures/convergence_error.png}
% \caption{Absolute error $|\widehat{V}_N - V_{\mathrm{BS}}|$ vs $N$ on log-log scale. Expected slope $\approx -1/2$.}
% \label{fig:convergence}
% \end{figure}

% \begin{figure}[H]
% \centering
% \includegraphics[width=0.85\textwidth]{figures/variance_reduction.png}
% \caption{Variance comparison: vanilla Monte Carlo vs antithetic variates. Antithetic should reduce variance and tighten confidence intervals.}
% \label{fig:antithetic}
% \end{figure}

\subsection{Interpretation (what to write)}
A strong Monte Carlo report should explicitly comment on:
\begin{itemize}
\item whether the Black--Scholes price lies inside the Monte Carlo 95\% CI,
\item how standard error changes with $N$ (decreasing roughly like $1/\sqrt{N}$),
\item how much variance reduction antithetic variates achieved (report a factor),
\item runtime vs accuracy trade-offs.
\end{itemize}

% ============================================================
\section{Discussion and Extensions}
\subsection{Summary}
We implemented a Monte Carlo engine for European option pricing under risk-neutral GBM, validated results against
Black--Scholes, and quantified uncertainty via confidence intervals. Antithetic variates reduced estimator variance,
improving efficiency while preserving unbiasedness.

\subsection{Natural Extensions}
This engine is a foundation for more advanced projects:
\begin{itemize}
\item \textbf{Path-dependent options:} Asian options, barrier options (requires time discretization).
\item \textbf{Greeks by Monte Carlo:} finite differences and likelihood ratio method.
\item \textbf{Control variates:} use Black--Scholes as a control to further reduce variance.
\item \textbf{Stochastic volatility:} Heston model simulation.
\end{itemize}

% ============================================================
\section*{References}
\addcontentsline{toc}{section}{References}

\begin{thebibliography}{9}

\bibitem{glasserman}
P. Glasserman,
\emph{Monte Carlo Methods in Financial Engineering},
Springer, 2004.

\bibitem{shreve}
S. E. Shreve,
\emph{Stochastic Calculus for Finance II: Continuous-Time Models},
Springer, 2004.

\bibitem{bjork}
T. Bj\"ork,
\emph{Arbitrage Theory in Continuous Time},
Oxford University Press, 3rd ed., 2009.

\end{thebibliography}

\end{document}
